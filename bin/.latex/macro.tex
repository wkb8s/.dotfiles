\usepackage{amsmath,amsfonts,bm,siunitx,here,url,listings,jlisting,graphicx,tcolorbox,pdfpages,hyperref,ascolorbox}
%tcolorboxとgraphicxを併用する際はグローバルでdvipdfmxを指定すること

% 余白
% \usepackage[margin=1.0in]{geometry}

%タイトル作成
\newcommand{\ttl}[1]{
  \title{{#1}}
  \author{Daiki Wakabayashi\thanks{Senior, Department of Information and Computer Science, 62021577}}
  \date{}
  \maketitle
}

%脚注の挿入
\newcommand{\fn}[1]{
  \protect\footnotemark\footnotetext{#1}
}

%2段組の際, 中央に線を入れる
\setlength{\columnseprule}{0.4pt}

%サブサブセクションまで目次に反映させる
\setcounter{tocdepth}{3}

%ソースコードの設定
\renewcommand{\lstlistingname}{Listing}
\lstset{
  basicstyle={\ttfamily},
  identifierstyle={\small},
  commentstyle={\smallitshape},
  keywordstyle={\small\bfseries},
  ndkeywordstyle={\small},
  stringstyle={\small\ttfamily},
  %↓コメントアウトでフレームを消せる
  frame={tb},
  breaklines=true,
  %columns=[l]{fullflexible},
  columns=fixed, %等幅設定
  basewidth=0.5em, %等幅設定
  numbers=left,
  xrightmargin=0zw,
  xleftmargin=3zw,
  numberstyle={\scriptsize},
  stepnumber=1,
  numbersep=1zw,
  lineskip=-0.5ex
}

%便利なアイコン達
\newcommand{\kaisetsu}%解説アイコン
{
\vskip.1\baselineskip %はじめに半分の空行
\noindent
\begin{tikzpicture}[scale=0.2, baseline=2.8pt]
\draw (3.3,1) node{\textgt{解 説}};
\draw[thick, rounded corners=3pt,] (0,0)--(6.5,0)--(6.5,2.2)--(0,2.2)--cycle;
\end{tikzpicture}\;\\}
\newcommand{\chui}{%注意アイコン
\vskip.1\baselineskip %はじめに半分の空行
\noindent
\begin{tikzpicture}[scale=0.2, baseline=2.8pt]
\fill (0,0)--(6.5,0)--(6.5,2.2)--(0,2.2);
\draw (3.3,1) node[white]{\textgt{注意!}};
\draw[thick] (0,0)--(6.5,0)--(6.5,2.2)--(0,2.2)--cycle;
\end{tikzpicture}\;\\}
\newcommand{\hosoku}{%補足アイコン
\vskip.1\baselineskip %はじめに半分の空行
\noindent
\begin{tikzpicture}[scale=0.2, baseline=2.8pt]
\draw (6,1) node{\textgt{補足}};
\fill (0,1)--(1,0)--(2,1)--(1,2)--cycle;
\fill[gray] (1,1)--(2,0)--(3,1)--(2,2)--cycle;
\fill (2,1)--(3,0)--(4,1)--(3,2)--cycle;
\fill (10,1)--(11,0)--(12,1)--(11,2)--cycle;
\fill[gray] (9,1)--(10,0)--(11,1)--(10,2)--cycle;
\fill (8,1)--(9,0)--(10,1)--(9,2)--cycle;
\end{tikzpicture}\;\\}
\newcommand{\answer}%証明アイコン
{
\vskip.1\baselineskip %はじめに半分の空行
\noindent
\begin{tikzpicture}[scale=0.2, baseline=2.8pt]
\draw (3.3,1) node{\textgt{解 答}};
\draw[double,thick,rounded corners=3pt,] (0,0)--(6.5,0)--(6.5,2.2)--(0,2.2)--cycle;
\end{tikzpicture}\;\\}


%ハイパーリンクの設定
%なるべく後ろに配置すること
\hypersetup{%
 setpagesize=false,%
 bookmarks=true,%
 bookmarksdepth=tocdepth,%
 bookmarksnumbered=true,%
 hidelinks,
 colorlinks=false,%
 pdftitle={},%
 pdfsubject={},%
 pdfauthor={},%
 pdfkeywords={}
}
